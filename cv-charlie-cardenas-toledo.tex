%%%%%%%%%%%%%%%%%%%%%%%%%%%%%%%%%%%%%%%%%
% CV Académico Profesional
% Charlie Alexander Cárdenas Toledo
%%%%%%%%%%%%%%%%%%%%%%%%%%%%%%%%%%%%%%%%%

\documentclass[11pt,a4paper,sans]{moderncv}

\moderncvstyle{classic}
\moderncvcolor{blue}

\usepackage[utf8]{inputenc}
\usepackage[scale=0.78]{geometry}
\usepackage{enumitem}

% Información Personal
\name{Charlie Alexander}{Cárdenas Toledo}
\title{Magíster en Ciencias y Tecnologías de la Computación}
\address{Loja}{Ecuador}{}
\email{charlie.act7@gmail.com}
\social[linkedin]{charlie-cardenas-toledo}
\social[github]{CharlieCardenasToledo}
\extrainfo{ORCID: \href{https://orcid.org/0000-0002-6658-7532}{0000-0002-6658-7532}}

\begin{document}

\makecvtitle

%----------------------------------------------------------------------------------------
\section{Formación Académica}
%----------------------------------------------------------------------------------------

\cventry{2020--2022}{Magíster en Ciencias y Tecnologías de la Computación}{Universidad Técnica Particular de Loja (UTPL)}{Loja, Ecuador}{}{
\textbf{Registro SENESCYT:} 1031-2022-2440152\\
\textbf{Tesis:} \textit{Uso de técnicas de machine learning en la identificación de patrones del perfil docente innovador}\\
\textbf{Director:} Luis Rodrigo Barba Guamán
}

\cventry{2014--2019}{Ingeniero en Sistemas Informáticos y Computación}{Universidad Técnica Particular de Loja (UTPL)}{Loja, Ecuador}{}{
\textbf{Registro SENESCYT:} 1031-2019-2122417
}

%----------------------------------------------------------------------------------------
\section{Experiencia Profesional}
%----------------------------------------------------------------------------------------

\cventry{May 2024--Presente}{Personal Docente -- Tiempo Completo}{Universidad Internacional del Ecuador (UIDE)}{Loja, Ecuador}{}{
\begin{itemize}[leftmargin=0.5cm, itemsep=0.15em]
    \item Profesor en programas de Ciencias de la Computación y Maestría
    \item Cursos: Ingeniería de Software, Sistemas de Gestión de Bases de Datos, Desarrollo Web
    \item Implementación de metodologías activas y herramientas de IA en enseñanza
    \item Desarrollo de sistemas de evaluación automatizados y plataformas CTF
\end{itemize}
}

\cventry{Oct 2022--Presente}{Personal Docente -- Tiempo Parcial}{Universidad Técnica Particular de Loja (UTPL)}{Loja, Ecuador (Híbrido)}{}{
\begin{itemize}[leftmargin=0.5cm, itemsep=0.15em]
    \item Docencia en Inteligencia Artificial y áreas relacionadas
    \item Modalidad híbrida combinando presencialidad y virtualidad
\end{itemize}
}

\cventry{Oct 2019--Oct 2023}{Técnico del Laboratorio de Innovación Docente}{Universidad Técnica Particular de Loja (UTPL)}{Loja, Ecuador}{}{
\begin{itemize}[leftmargin=0.5cm, itemsep=0.15em]
    \item Responsable técnico del Laboratorio de Investigación e Innovación Docente (LiiD)
    \item Gestión de proyectos de innovación educativa
    \item Desarrollo de herramientas y plataformas para apoyo docente
    \item Email institucional: cacardenas7@utpl.edu.ec
\end{itemize}
}

\cventry{Abr 2015--Sept 2019}{Desarrollador de Software}{Universidad Técnica Particular de Loja (UTPL)}{Laboratorio de Inteligencia Artificial, Loja, Ecuador}{}{
\begin{itemize}[leftmargin=0.5cm, itemsep=0.15em]
    \item Desarrollo de soluciones de software en el Laboratorio de IA
    \item Participación en proyectos de investigación aplicada
    \item Implementación de sistemas inteligentes y arquitecturas multi-agente
\end{itemize}
}

%----------------------------------------------------------------------------------------
\section{Producción Científica}
%----------------------------------------------------------------------------------------

\subsection{Publicaciones (4 trabajos indexados)}

\cvitem{2026}{\textbf{CARE+: A Methodology for AI-Enhanced Learning that Develops 21st Century Competencies}}
\cvitem{}{Con F.M. Soto Guerrero y M.I. Loaiza}
\cvitem{}{Book chapter | DOI: 10.1007/978-3-032-08366-1\_8}

\cvitem{2023}{\textbf{A Topic Modeling Approach to Analyze Teaching Innovation Projects}}
\cvitem{}{Con S. Rosales y R. Reátegui}
\cvitem{}{2023 Fourth International Conference on Information Systems and Software Technologies (ICI2ST)}
\cvitem{}{DOI: 10.1109/ici2st62251.2023.00014}

\cvitem{2022}{\textbf{Use of clustering techniques in the search for innovative teacher characteristics}}
\cvitem{}{Con L. Barba-Guamán, M.I. Loaiza y P. Valdiviezo-Diaz}
\cvitem{}{2022 17th Iberian Conference on Information Systems and Technologies (CISTI)}
\cvitem{}{DOI: 10.23919/cisti54924.2022.9820569}

\cvitem{2020}{\textbf{Diseño e implementación de una arquitectura de comunicación entre dispositivos tecnológicos basados en comunidades de agentes para un salón inteligente}}
\cvitem{}{Book chapter}

\subsection{Identificadores Académicos}

\cvitem{ORCID}{\href{https://orcid.org/0000-0002-6658-7532}{0000-0002-6658-7532}}
\cvitem{ResearchGate}{RG Score: 6.6 | 4 Citas | h-index: 1}

%----------------------------------------------------------------------------------------
\section{Proyectos de Investigación}
%----------------------------------------------------------------------------------------

\cvitem{2018--2023}{\textbf{Proyecto Ascendere -- UTPL}}
\cvitem{}{Análisis de datos educativos mediante machine learning (clustering, topic modeling) para identificación de perfiles docentes innovadores.}
\cvitem{}{Resultó en tesis de maestría y 2 publicaciones en conferencias internacionales.}

\cvitem{2019--2020}{\textbf{Sistema SaCI (Smart Classroom) -- UTPL}}
\cvitem{}{Desarrollo de arquitectura multi-agente para aulas inteligentes con IoT.}
\cvitem{}{Publicación en book chapter (2020) e implementación funcional en UTPL.}

%----------------------------------------------------------------------------------------
\section{Áreas de Especialización}
%----------------------------------------------------------------------------------------

\cvitem{}{\textbf{IA en Educación}}
\cvitem{}{Metodologías de aprendizaje aumentado por IA, desarrollo de competencias del siglo XXI}

\cvitem{}{\textbf{Machine Learning}}
\cvitem{}{Topic modeling (LDA), clustering, análisis de datos educativos}

\cvitem{}{\textbf{Sistemas Inteligentes}}
\cvitem{}{Arquitecturas multi-agente, IoT educativo, aulas inteligentes}

\cvitem{}{\textbf{Docencia}}
\cvitem{}{Programación (Python, JavaScript/TypeScript), Ingeniería de Software, Bases de Datos, Desarrollo Web}
\cvitem{}{Metodologías: Aprendizaje basado en proyectos, aula invertida, gamificación, integración de IA}

%----------------------------------------------------------------------------------------
\section{Competencias Técnicas}
%----------------------------------------------------------------------------------------

\cvitem{Lenguajes}{JavaScript, TypeScript, Python, R, SQL}

\cvitem{Frameworks}{\textbf{Frontend:} Angular, React, Next.js | \textbf{Backend:} NestJS, Flask, Node.js | \textbf{ML:} Scikit-learn, Pandas, NumPy}

\cvitem{Bases de Datos}{PostgreSQL, MySQL, MongoDB, Firebase}

\cvitem{Herramientas}{Git/GitHub, Docker, Jupyter, LaTeX, ChatGPT, Gemini, Claude Code}

%----------------------------------------------------------------------------------------
\section{Desarrollo de Software}
%----------------------------------------------------------------------------------------

\cvitem{Portafolio}{\href{https://github.com/CharlieCardenasToledo}{github.com/CharlieCardenasToledo} -- 28+ repositorios públicos}

\cvitem{Destacados}{
\begin{itemize}[leftmargin=0.5cm, itemsep=0.1em]
    \item \textbf{evaluaciones-is/sgbd:} Sistemas de evaluación automatizados
    \item \textbf{CTFd-UIDE:} Plataforma CTF para enseñanza de ciberseguridad
    \item \textbf{prompts-for-edu:} Colección de prompts educativos con IA
    \item \textbf{UIDE-Manager:} Sistema de gestión académica
\end{itemize}
}

%----------------------------------------------------------------------------------------
\section{Idiomas}
%----------------------------------------------------------------------------------------

\cvitemwithcomment{Español}{Nativo}{}
\cvitemwithcomment{Inglés}{B1 (Intermedio)}{Lectura y comprensión de documentación técnica}

%----------------------------------------------------------------------------------------
\section{Información Adicional}
%----------------------------------------------------------------------------------------

\cvitem{Filosofía}{\textit{Educador enfocado en transformar el potencial de estudiantes en habilidades reales, combinando teoría con práctica mediante metodologías activas, proyectos reales y tecnologías actuales. Formando no solo técnicos, sino solucionadores creativos de problemas.}}

%----------------------------------------------------------------------------------------
\section{Referencias}
%----------------------------------------------------------------------------------------

\cvitem{}{\textbf{Luis Rodrigo Barba Guamán, Ph.D}}
\cvitem{}{Director de Maestría en Inteligencia Artificial Aplicada}
\cvitem{}{Universidad Técnica Particular de Loja}
\cvitem{}{Director de tesis de maestría}
\cvitem{}{\href{https://www.linkedin.com/in/luis-rodrigo-barba-guaman-ph-d-383b3a6}{LinkedIn}}

\cvitem{}{\textbf{Fernanda Soto, Magíster}}
\cvitem{}{Docente}
\cvitem{}{Universidad Técnica Particular de Loja}
\cvitem{}{Supervisora directa}
\cvitem{}{\href{https://www.linkedin.com/in/fernanda-soto-04725353/}{LinkedIn}}

\cvitem{}{\textbf{Ana Gabriela Vargas Castillo, Magíster}}
\cvitem{}{Gestor de Reconocimiento de Estudios}
\cvitem{}{Universidad Técnica Particular de Loja}
\cvitem{}{Colega de equipo}
\cvitem{}{\href{https://www.linkedin.com/in/ana-gabriela-vargas-castillo/}{LinkedIn}}

\cvitem{}{\textbf{Paola Abad, Magíster}}
\cvitem{}{Docente Universitaria}
\cvitem{}{Universidad Internacional del Ecuador (UIDE)}
\cvitem{}{Colega de equipo}
\cvitem{}{\href{https://www.linkedin.com/in/paolaabadjaramillo/}{LinkedIn}}

\end{document}
